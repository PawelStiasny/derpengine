\documentclass[12pt,a4paper]{article}

\usepackage{graphicx}

\usepackage[utf8]{inputenc}
\usepackage[polish]{babel}
\usepackage{polski}

\title{Mech - zadanie zaliczeniowe GKOM}
\author{Paweł Stiasny}
\date{Semestr 12Z}

\begin{document}
\begin{titlepage}
\maketitle

%\section{Ogólny opis rozwiązania}
\begin{abstract}
Projekt ma realizować wizualizację kroczącego Mecha przypominajacego maszyny z
gry Mech Warrior.  Będzie on posiadał widoczne uzbrojenie i dodatkowe efekty
zwiększające realizm animacji.

Rozwiązanie jest oparte o graf sceny zaimplementowany w C++.  Szczegółowy jego
projekt jest opisany w sekcji~\ref{sec:klasy}.  Może być on względnie łatwo
rozszerzany przy dalszych pracach nad projektem.

Zdecydowałem się zrealizować zadanie w sposób maksymalnie przenośny między
platformami.  Szczegóły dotyczące realizacji tego założenia opisane są w
sekcji~\ref{sec:narz}.
\end{abstract}
\end{titlepage}

\section{Wygląd sceny}
Scena ma przedstawiać kroczącego (z ruchomymi stawami) Mecha na płaskim
terenie.  Na maszynę nałożona będzie tekstura opancerzenia.  Oświetlenie
pochodzi od słońca.

Kamera będzie podążać za Mechem, ewentualnie zmieniając kąt patrzenia w
zależności od tego, czy jest on w ruchu.

Jako dodatkowy efekt dodający realizmu animacji, ze stawów kończyn mogą
wydobywać się iskry.

\section{Hierarchia klas}
\label{sec:klasy}
\subsection{GraphNode}
Rysowanie obiektów oparte jest o graf sceny.  Obiekty sceny reprezentowane są
przez podklasy klasy \texttt{GraphNode} realizującej wzorzec Kompozyt.
\texttt{GraphNode} realizuje przesunięcie i obrót obiektu, zarządza
renderowaniem siebie i dzieci w metodzie szablonowej \texttt{render()} oraz
może być użyty bezpośrednio jako konter na obiekty.

\begin{center}
%\centering
% \leavevmode
\includegraphics[width=190pt]{latex/classGraphNode__inherit__graph}
\end{center}

\subsection{Animation}
Klasa Animation umożliwia dogodne tworzenie abstrakcji animacji w scenie.
Udostępnia ona tylko metodę \texttt{update(float~timestep)}, a jej podklasy
są~dodawane do globalnej listy aktywnych animacji.

\begin{center}
%\centering
% \leavevmode
\includegraphics[width=202pt]{latex/classAnimation__inherit__graph}
\end{center}

\section{Użyte biblioteki i narzędzia}
\label{sec:narz}
\subsection{Biblioteka SDL}
Jako biblioteki pomocniczej zdecydowałem się użyć Simple DirectMedia Layer
zamiast proponowanego glut.  Spełnia ona podobne zadania, ale jest bardziej
elastyczna i w odróżnieniu od glut jest aktywnie rozwijana.  Obie biblioteki
pozwalają na łatwe przenoszenie aplikacji między platformami.

\subsection{Budowanie projektu za pomocą SCons}
Do kompilowania używam pythonicznego systemu budowania SCons.  Narzędzie to jest
przenośne i umożliwia współpracę z docelowymi narzędziami Microsoftu.

\end{document}
