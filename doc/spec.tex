\documentclass[12pt,a4paper]{article}

\usepackage[utf8]{inputenc}
\usepackage[polish]{babel}
\usepackage{polski}

\title{Mech - zadanie zaliczeniowe GKOM}
\author{Paweł Stiasny}
\date{Semestr 12Z}

\begin{document}
\begin{titlepage}
\maketitle

%\section{Ogólny opis rozwiązania}
\begin{abstract}
Projekt ma realizować wizualizację kroczącego Mecha przypominajacego maszyny z
gry Mech Warrior.  Będzie on posiadał widoczne uzbrojenie i dodatkowe efekty
zwiększające realizm animacji.  Na obiekt zostanie nałożona tekstura
opancerzenia.

Rozwiązanie jest oparte o graf sceny zaimplementowany w C++.  Szczegółowy jego
projekt jest opisany w sekcji~\ref{sec:klasy}.  Może być on względnie łatwo
rozszerzany przy dalszych pracach nad projektem.

Zdecydowałem się zrealizować zadanie w sposób maksymalnie przenośny między
platformami.  Szczegóły dotyczące realizacji tego założenia opisane są w
sekcji~\ref{sec:narz}.
\end{abstract}
\end{titlepage}

\section{Hierarchia klas}
\label{sec:klasy}
\subsection{GraphNode}
Rysowanie obiektów oparte jest o graf sceny.  Obiekty sceny reprezentowane są
przez podklasy klasy GraphNode realizującej wzorzec Kompozyt.  GraphNode
realizuje przesunięcie i obrót obiektu, zarządza renderowaniem siebie i dzieci w
metodzie szablonowej render oraz może być użyty bezpośrednio jako konter na
obiekty.
% Tu schematy itp

\subsection{Animation}

\section{Użyte biblioteki i narzędzia}
\label{sec:narz}
\subsection{Biblioteka SDL}
Jako biblioteki pomocniczej zdecydowałem się użyć SDL zamiast proponowanego
glut.  Spełnia ona podobne zadania, ale jest bardziej elastyczna i w odróżnieniu
od glut jest aktywnie rozwijana.  Obie biblioteki pozwalają na łatwe
przenoszenie aplikacji między platformami.

\subsection{Budowanie projektu za pomocą SCons}
Do kompilowania używam Pythonicznego systemu budowania SCons.  Narzędzie to jest
przenośne i umożliwia współpracę z docelowymi narzędziami Microsoftu.

\end{document}
